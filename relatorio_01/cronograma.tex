\section{Agenda de atividades}

\begin{itemize}
  \item 28 e 29/03 - Conclus�o da se��o Poisson 1D;
  \item 30 e 31/03 - Iniciar reda��o da se��o de revis�o em Turbul�ncia;
  \item 01/04 - 10/04 - Implementa��o da equa��o do calor trasiente 1D e 2D + relat�rio Turbul�ncia;
  \item 11/04 - 30/04 - Implementa��o da equa��o de Stokes 2D e 3D + relat�rio Turbul�ncia;
\end{itemize}

AGENDA ABRIL/2017

* FAZER DETALHES DIFEREN�AS FINITAS NO RELAT�RIO\\
* ALTERAR CONDI��ES DE CONTORNO\\
* TRABALHAR NA VISUALI��O DE RESULTADOS (VTK ou OUTRO)\\
* IMPLEMENTAR DIFERENTES ORDENS EM PETSC\\
* CRIAR FUN��O PLOTAR ORDEM ERRO\\
* IMPLEMENTAR EQUA��O CALOR TRANSIENTE\\
* IMPLEMENTAR EQUA��O STOKES\\
* UM EXEMPLO EM MALHA N�O ESTRUTURADA

MAIO 

* IMPLEMENTAR AS EQUA��ES EM VOLUMES FINITOS\\

Reuni�o Fabricio 07/04/2017
* Empregar t�cnicas implicitas na equa��o do calor

  