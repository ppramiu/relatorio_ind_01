\begin{itemize}
	\item Introdu��o
		\subitem Problem�tica;
		\subitem Motiva��o;
		\subitem Justificativas;
		\subitem Objetivos;
	\item Fundamentos e Revis�o Bibliogr�fica
		\subitem Aspectos f�sicos;
		\subitem Aspectos matem�ticos;
		\subitem Vis�o geral de modelagem turbulenta;
		\subitem Revis�o Bibliogr�fica;
	\item Modelagem da turbul�ncia por meio de TLES
		\subitem Aspectos f�sicos e matem�ticos;
		\subitem Aspectos num�ricos;
	\item Acoplamento da hidrodin�mica e transporte com o modelo de TLES
		\subitem Hidrodin�mica (N-S);
		\subitem Transporte e DPM;
		\subitem Abras�o-Eros�o;
	\item Proposta de Pesquisa
		\subitem Termo de regulariza��o;
		\subitem Tensor submalha;
		\subitem Solu��o num�rica n�o-oscilat�ria - tvd?
		\subitem Estudo comparitivo de alguns modelos;
		\subitem
	\item Verifica��o, Valida��o e Discuss�o Resultados
		\subitem Experimentos num�rico-computacionais;
		\subitem Case i) Cubo;
		\subitem Case ii) Vertedouro;
	\end{itemize}
